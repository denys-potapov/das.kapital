\parcont{}  %% абзац починається на попередній сторінці
\index{ii}{0044}  %% посилання на сторінку оригінального видання
формі скарбів, що буває також і в простій товаровій циркуляції, скоро
зовнішні обставини переривають перехід $Т — Г$ в $Г — Т$. Це є вимушене
утворення скарбів. Отже, в даному разі гроші мають форму бездіяльного
капіталу, лятентного грошового капіталу\footnote{
Термін „лятентний“ взято з фізичного уявлення про лятентне тепло, що
його тепер згідно з теорією перетворення енергії майже облишено. Тому Маркс
у третьому відділі (пізніша редакція) замість нього подає термін, узятий з уявлення
про потенціяльну енерґію: „потенціяльний“, або за аналогією з віртуальними швидкостями
д’Алямбера, „віртуальний капітал“. \emph{Ф.~Е.}
}. Однак покищо ми не будемо
далі зупинятись на цьому.

Але в обох випадках залишення грошового капіталу в його стані
грошей з’являється як наслідок перерваного руху, все одно, чи та перерва
доцільна, чи недоцільна, добровільна чи недобровільна, відповідає функціям
чи суперечить їм.

\subsection{Акумуляція і репродукція в поширеному маштабі}

Що відношення, в яких можливе поширення продукційного процесу,
визначається не довільно, а диктується технікою, то реалізована додаткова
вартість, хоча б її й призначалось для капіталізації, часто лише по
кількох повторюваних кругобігах доходить такого розміру (отже, мусить
бути нагромаджена до такого розміру), коли вона справді може функціонувати
як додатковий капітал, або ввійти в кругобіг капітальної вартости,
що процесує. Отже, додаткова вартість закам’яніває в скарб і в цій
формі становить лятентний грошовий капітал. Лятентний тому, що вона не
може діяти як капітал, поки вона лишається в грошовій формі. Отже,
утворення скарбу з’являється тут як момент, що входить у процес капіталістичної
акумуляції, супроводить його, але разом з тим і посутньо
відрізняється від нього. Бо утворення лятентного грошового капіталу
не поширює самого процесу репродукції. Лятентний грошовий
капітал тому утворюється тут, що капіталістичний продуцент не може
безпосередньо поширювати свою продукцію. Коли він продає свій додатковий
продукт продуцентові золота або срібла, що подає в циркуляцію
нове золото або срібло, або — що сходить на те саме — купцеві, що
за деяку частину національного додаткового продукту довозить
з-за кордону додаткове золото або срібло, то його лятентний грошовий
капітал являє приріст національного золотого або срібного скарбу. В усіх
інших випадках, напр., ті 78\pound{ ф. стерл.}, що в руках покупця були засобом
циркуляції, в руках капіталіста набирають лише форму скарбу;
отже, постає лише інший розподіл національного золотого або срібного
скарбу.

Коли в оборудках нашого капіталіста гроші функціонують як засіб виплати
(так що покупець платить за товари лише через більш або менш довгий
строк), то додатковий продукт, призначений для капіталізації, перетворюється
не на гроші, а на боргові вимоги, у титули власности
на еквівалент, що вже або є в покупця, або лише передбачається. Цей
\parbreak{}  %% абзац продовжується на наступній сторінці
