\parcont{}  %% абзац починається на попередній сторінці
\index{ii}{0268}  %% посилання на сторінку оригінального видання
само метаморфоза індивідуального капіталу, його оборот, є ланка в кругобігу
суспільного капіталу.

Цей сукупний процес охоплює так само і продуктивне споживання
(безпосередній процес продукції) разом з перетвореннями форми (обмінами,
коли розглядати справу з речового боку), що упосереднюють його,
і особисте споживання з перетвореннями форми або обмінами, що упосереднюють
це споживання. Він охоплює, з одного боку, перетворення
змінного капіталу на робочу силу, а значить, і введення робочої сили в
капіталістичний процес продукції. Тут робітник виступає як продавець
свого товару, робочої сили, а капіталіст як покупець її. Але, з другого
боку, продаж товару включає й купівлю його робітничою клясою, отже, її
особисте споживання. Тут робітнича кляса виступає як покупець, а капіталісти
— як продавці товарів робітникам.

Циркуляція товарового капіталу включає й циркуляцію додаткової
вартости, а значить, і купівлі й продажі, що ними капіталісти упосереднюють
своє особисте споживання, споживання додаткової вартости.

Кругобіг індивідуальних капіталів, розглядуваних у їхньому з’єднанні
в суспільний капітал, отже, кругобіг, розглядуваний в його цілості, охоплює
не лише циркуляцію капіталу, а й загальну товарову циркуляцію.
Ця остання може первісно складатись лише з двох складових частин:
1) власне кругобігу капіталу і 2) кругобігу товарів, що входять в особисте
споживання, отже, товарів, що на них робітник витрачає свою заробітну
плату, а капіталіст — свою додаткову вартість (або частину своєї
додаткової вартости). В усякому разі кругобіг капіталу охоплює й циркуляцію
додаткової вартости, оскільки вона становить частину товарового
капіталу, а також охоплює і перетворення змінного капіталу на робочу
силу, виплату заробітної плати. Але витрачання цієї додаткової вартости
та заробітної плати на товари не становить жодної ланки циркуляції
капіталу, не зважаючи на те, що принаймні витрачання заробітної плати
зумовлює цю циркуляцію.

В І книзі проаналізовано капіталістичний процес продукції і як окремий
процес і як процес репродукції: продукцію додаткової вартости
і продукцію самого капіталу. Зміни форми та речовин, що їх проробляє
капітал у сфері циркуляції, ми припустили як передумову, що на
ній не зупинялись далі. Отже, ми припускали, що капіталіст, з одного
боку, продає продукт за його вартістю, а з другого, знаходить у сфері
циркуляції речові засоби продукції, потрібні для того, щоб відновити
процес або безупинно провадити його. Єдиним актом у сфері циркуляції,
що на ньому нам довелось там зупинитись, був акт купівлі та продажу
робочої сили як основної умови капіталістичної продукції.

В першому відділі цієї II книги ми розглядали різні форми, що їх
набирає капітал у своєму кругобігу, та різні форми самого цього кругобігу.
До робочого часу, розглянутого в І книзі, тепер долучається час
циркуляції.

В другому відділі ми розглядали кругобіг капіталу як періодичний
процес, тобто як оборот капіталу. Ми показали, з одного боку, як різні
\index{ii}{0269}  %% посилання на сторінку оригінального видання
складові частини капіталу (основний і обіговий) пророблюють кругобіг форм
в різні періоди часу й різним способом; з другого боку, ми дослідили обставини,
що ними зумовлюється різний протяг робочого періоду й періоду
циркуляції. Ми показали, як впливає період кругобігу й різне відношення
його складових частин на розмір самого продукційного процесу і на
річну норму додаткової вартости. В дійсності, коли в першому відділі
розглядалось переважно послідовні форми, що їх у своєму кругобігу капітал
постійно набирає й скидає, то в другому відділі ми розглянули, як
у межах цього руху й послідовности форм капітал даної величини одночасно,
хоч і в змінному розмірі, поділяється на різні форми — на продуктивний
капітал, грошовий капітал і товаровий капітал, так, що ці
форми не лише чергуються одна з однією, але різні частини сукупної
капітальної вартости постійно одна поряд однієї перебувають і функціонують
у цих різних станах. Саме грошовий капітал при цьому виявив
особливість, яка не виявлялась в книзі першій. Ми виявили ті певні закони,
що згідно з ними різні величиною складові частини даного капіталу,
відповідно до умов обороту, постійно мусять авансуватись і відновлюватись
у формі грошового капіталу для того, щоб підтримувати
продуктивний капітал даного розміру в безперервному функціонуванні.

Але і в першому і в другому відділі мова була завжди тільки про індивідуальний
капітал, про рух усамостійненої частини суспільного капіталу.

Але кругобіги індивідуальних капіталів переплітаються один з одним,
являють передумову і зумовлюють один одного і саме в цьому сплетінні
й становлять рух сукупного суспільного капіталу. Як при простій товаровій
циркуляції уся метаморфоза одного товару виступала як ланка
ряду метаморфоз товарового світу, так тепер метаморфоза індивідуального
капіталу виступає як ланка ряду метаморфоз суспільного капіталу. Але
коли проста циркуляція товарів зовсім не включає неодмінно циркуляції
капіталу, — бо товарова циркуляція може відбуватись на основі некапіталістичної
продукції, — то кругобіг сукупного суспільного капіталу включає,
як уже зазначено, і товарову циркуляцію, що не входить в кругобіг
індивідуального капіталу, тобто включає циркуляцію товарів, які не є
капітал.

Тепер ми повинні розглянути процес циркуляції (а він у своїй сукупності
є форма процесу репродукції) індивідуальних капіталів, як складових
частин сукупного суспільного капіталу, отже, розглянути процес
циркуляції цього суспільного сукупного капіталу.

\subsection{Роля грошового капіталу}

[Хоч дальший виклад належить до пізнішої частини цього відділу,
все ж ми зараз дослідимо це, тобто грошовий капітал, розглядуваний
як складова частина суспільного сукупного капіталу].

При розгляді обороту індивідуального капіталу грошовий капітал виявив
себе з двох боків.
