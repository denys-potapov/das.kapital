\parcont{}  %% абзац починається на попередній сторінці
\index{ii}{0275}  %% посилання на сторінку оригінального видання
вартости, капіталістичний фармер, відрізняється від того, хто її просто
привласнює.

Капіталістичний характер фізіократичної системи ще за доби її розквіту
викликав опозицію, з одного боку, Лінґе й Маблі, а з другого —
захисників дрібного вільного землеволодіння.
\pfbreak
Те, що А.~Сміс у своїй аналізі процесу репродукції робить крок
назад\footnote{
„Капітал“, т. І, розділ XXII, 2, прим. 32.
}, впадає на очі то більше, що він взагалі не лише далі опрацьовує
правильну аналізу Кене, напр., узагальнюючи його „avances primitives“
і „avances annuelles“ в „основний“ капітал та „обіговий“
капітал\footnote{
І тут йому розчистили шлях деякі фізіократи, насамперед Тюрґо. Останній
уже частіше, ніж Кене та інші фізіократи, вживає слово „капітал“ замість avances
і ще більше ототожнює avances або capitaux мануфактуристів із avances або
capitaux фармерів. Напр., „Так само як ці останні (підприємці-мануфактуристи),
вони (фармери, тобто капіталістичні орендарі) повинні одержувати, крім повернених
капіталів“ і~\abbr{т. ін.} („Comme eux les entrepreneurs-manufacturiers), ils (les fermiers)
doivent recueillir outre la rentrée de leurs capitaux etc.“ — Turgot, Oeuvres, éd.
Daire, Paris, 1844. Tome I, p. 40).
}, але подекуди й зовсім допускається фізіократичних помилок.
Напр., щоб довести, що фармер продукує більшу вартість, ніж яка інша
відміна капіталістів, він каже: „Жоден інший капітал однакової величини
не пускає в рух більшої маси продуктивної праці, ніж капітал фармера.
Не лише його челядь, але й робоча худоба його складається з продуктивних
робітників“. (Приємний комплімент для челяді). „В хліборобстві
поряд людей працює також природа; і хоч її \so{праця не коштує
жодних витрат}, все ж її продукт має свою \so{вартість, цілком
так само, як продукт праці найдорожчих робітників.}
Найважливіші операції в хліборобстві, здається, спрямовано не так на те,
щоб підвищити природну родючість, хоч вони спричиняють і це, — як
на те, щоб повернути її на продукцію найкорисніших для людини рослин.
Поле, заросле бур’яном, досить часто дає таку ж саму велику
кількість рослинности, як і найкраще оброблений виноградник або
нива. Насадження рослин і культивування часто мають більший вплив на
реґулювання, ніж на відживлення активної родючости природи. Після
того як усю працю обробітку вивершено, природі припадає ще чимала
частина роботи. Отже, робітники та робоча худоба (!), зайняті в хліборобстві,
не лише репродукують, на зразок мануфактурних робітників,
вартість, рівну власному їхньому споживанню та капіталові, що їх уживає,
до праці, плюс зиск капіталіста: вони репродукують куди більшу вартість.
Крім капіталу фармера та всього його зиску вони реґулярно репродукують
ще й ренту землевласника. Ренту можна розглядати, як продукт
природних сил, що користування ними землевласник позичає орендареві.
Вона більша або менша, залежно від припушуваного рівня цих сил,
тобто залежно від припущеної природної або штучно досягненої родючости
ґрунту. Вона — продукт природи, який лишається по відліченні
\parbreak{}  %% абзац продовжується на наступній сторінці
