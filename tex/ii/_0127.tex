\parcont{}  %% абзац починається на попередній сторінці
\index{ii}{0127}  %% посилання на сторінку оригінального видання
мости, тунелі, віадуки тощо, являє приклад того, що можна назвати
віковим зношуванням. А швидше й помітніше зневартнення, відшкодуване
протягом коротких переміжків часу ремонтом і заміщенням, є подібне
до періодичних неправильностей. У витрати на річний ремонт заводиться
й полагодження тієї випадкової шкоди, що її зазнають час від часу
зовнішні частини навіть довготриваліших споруд; але й незалежно від
такого ремонту, час не минає для них безслідно, і хоч як далекий той момент,
коли стан цих будов потребуватиме перебудувати їх наново, а все ж
мусить він надійти. В усякому разі щодо фінансової та економічної сторони
цей момент може бути дуже віддалений, щоб його брати на увагу
в практичних обчисленнях“ (Lardner, 1. c., 38, 39).

Це має силу до всіх таких споруд вікової тривалости, що в них,
отже, не доводиться поступінно, рівнобіжно з їхнім зношенням, заміщувати
авансований на них капітал, а доводиться переносити на ціну продукту
лише щорічні пересічні витрати на підтримання і ремонт.

Хоча — як ми бачили — більшість грошей, які щороку або навіть
через коротший час повертаються на заміщення зношуваного основного
капіталу, знову перетворюються на натуральну форму цього капіталу,
проте, кожному поодинокому капіталістові потрібен фонд амортизації для тієї
частини основного капіталу, що для неї лише по багатьох роках надходить час
репродукції, і її треба тоді цілком заміщувати. Значна складова частина основного
капіталу вже в наслідок своїх властивостей виключає часткову репродукцію.
Крім того, там, де частинна репродукція відбувається таким способом, що
через короткі переміжки до зневартненого складу додається новий, то, щоб це
заміщення було можливе, потрібне попереднє грошове нагромадження
в більших або менших розмірах, залежно від специфічного характеру
даної галузі продукції. Для цього досить не якої завгодно суми грошей,
а грошової суми певних розмірів.

Коли ми розглянемо цю справу, припускаючи лише просту грошову
циркуляцію, лишаючи цілком осторонь кредитову систему, що про неї
мова буде далі, то механізм руху такий: в першій книзі (розділ II, 3а)
показано, що коли одна частина наявних у суспільстві грошей завжди
лежить без діла як скарб, а друга функціонує як засіб циркуляції, зглядно
як безпосередній резервний фонд для грошей, що безпосередньо циркулюють,
то постійно змінюється пропорція, що в ній уся маса грошей
розподіляється на скарб і на засоби циркуляції. В нашому прикладі
гроші — що їх досить великий капіталіст повинен нагромадити як скарб
чималих розмірів, — підчас закупу основного капіталу разом пускається
в циркуляцію. Потім вони знову сами собою розпадаються в суспільстві
на засоби циркуляції та скарб. За допомогою амортизаційного фонду,
куди, як до свого вихідного пункту, повертається вартість основного капіталу
в міру його зношування, частина грошей, що циркулюють, знову
утворює скарб — на більший або менший час — в руках того самого капіталіста,
що від нього підчас закупу основного капіталу віддалився його
скарб, перетворившись на засіб циркуляції. Отже, ми маємо повсякчас
змінний розподіл наявного в суспільстві скарбу, що по черзі функціонує
\parbreak{}  %% абзац продовжується на наступній сторінці
