\parcont{}  %% абзац починається на попередній сторінці
\index{ii}{0037}  %% посилання на сторінку оригінального видання
ряду, що є заміщення $т$. Те саме буває й тоді, коли акт $Т' — Г'$ не
вдається провести, або коли можна продати лише частину $Т'$.

Ми бачили, що $т — г — т$, як циркуляція доходу капіталіста, входить
в циркуляцію капіталу лише доти, доки $т$ є частина вартости $Т'$, капіталу
в його функціональній формі товарового капіталу; але, усамостійнившися
через $г — т$, отже, в цілій формі $т — г — т$, циркуляція доходу не входить
у рух капіталу, авансованого капіталістом, хоч і походить від нього.
Вона зв’язана з циркуляцією авансованого капіталу остільки, оскільки
існування капіталу має собі за передумову існування капіталіста, а це
останнє зумовлене його споживанням додаткової вартости.

У загальній циркуляції $Т'$, прим., пряжа, функціонує лише як товар;
але як момент циркуляції капіталу функціонує вона як товаровий
капітал, форма, що її капітальна вартість по черзі то набирає, то скидає.
Після того, як пряжу продано покупцеві, вона виходить із процесу
кругобігу того капіталу, що його продукт вона являє, а, проте, все ще
перебуває як товар у сфері загальної циркуляції. Циркуляція цієї товарової
маси все ще триває далі, хоч вона перестала вже являти момент
у самостійному кругобігу капіталу прядуна. Тому справжня остаточна
метаморфоза товарової маси, що її пускає в циркуляцію капіталіст, акт
$Т — Г$, остаточний вихід її в сферу споживання, може бути цілком відокремлений
у часі й просторі від тієї метаморфози, що в ній ця товарова
маса функціонує як товаровий капітал цього капіталіста. Ту саму метаморфозу,
що відбулася в сфері циркуляції капіталу, треба ще зробити
в сфері загальної циркуляції.

Справа зовсім не змінюється від того, що пряжа знову входить
у кругобіг іншого промислового капіталу. Загальна циркуляція охоплює
і навзаєм сплетені кругобіги різних самостійних частин суспільного капіталу,
тобто сукупність поодиноких капіталів, і циркуляцію вартостей, поданих
на ринок не як капітал, тобто вартостей, що входять у сферу особистого
споживання.

Відношення між кругобігом капіталу, оскільки він є частина загальної
циркуляції та оскільки він є член самостійного кругобігу, виявляється
далі, коли ми розглядаємо циркуляцію $Г' \deq{} Г \dplus{} г$. $Г$, як грошовий капітал,
і далі продовжує кругобіг капіталу; $г$, як витрачання доходу ($г — т$),
входить у загальну циркуляцію, але виходить із кругобігу капіталу. У цей
останній кругобіг увіходить лише та частина $г$, яка функціонує як додатковий
грошовий капітал. У $т — г — т$ гроші функціонують лише як
монета; мета цієї циркуляції — особисте споживання капіталіста. Кретинізм
вульґарної економії характеризує та обставина, що цю циркуляцію, яка
не входить у кругобіг капіталу — циркуляцію тієї частини новоспродукованої
вартости, що її споживається як дохід — вона видає за характеристичний
кругобіг капіталу.

У другій фазі, $Г — Т$, знову перед нами капітальна вартість
$Г \deq{} П$ (вартості продуктивного капіталу, який тут розпочинає кругобіг
промислового капіталу), позбавлена додаткової вартости, отже, такої
самої величини вартости, як на першій стадії кругобігу грошового
\parbreak{}  %% абзац продовжується на наступній сторінці
