\parcont{}  %% абзац починається на попередній сторінці
\index{ii}{0276}  %% посилання на сторінку оригінального видання
або покриттю всього того, що можна вважати за продукт рук людських.
Вона рідко коли менша, ніж чверть, і часто більша, ніж третина цілого
продукту. Жодна однакова маса продуктивної праці, застосована в мануфактурі,
ніколи не може зумовити такої великої репродукції. В мануфактурі
природа не робить нічого, людина — все; а репродукція завжди
мусить бути пропорційна потужності аґентів, що її переводять. Тому
капітал, вкладений у хліборобство, не лише пускає в рух більшу масу
продуктивної праці, ніж якийбудь інший, однаковий величиною капітал,
застосований у мануфактурі, але, порівняно з масою зайнятої ним продуктивної
праці, він додає куди більшу вартість до річного продукту
землі та праці даної країни, — до цього справжнього багатства і доходу
її жителів“. (Кн.~II, розд. 5, стор. 242).

А.~Сміс каже в II книзі, розд. І; „Вся вартість засівного матеріялу
теж є власне основний капітал“. Отже, тут капітал \deq{} капітальній вартості;
він існує в „основній“ формі. „Хоч засівний матеріял завжди переходить
з поля до комори й навпаки, він ніколи не змінює свого власника, а тому
в дійсності не циркулює. Фармер здобуває свій зиск не через його продаж,
а через його приріст“, (ст. 186). Обмеженість тут у тому, що Сміс
не бачить, як то вже бачив Кене, що вартість сталого капіталу знову
з’являється в відновленій формі, отже, не бачить важливого моменту
процесу репродукції, а бачить лише ще одну ілюстрацію — та до того ж
і фалшиву — свого відрізнювання між обіговим капіталом і основним
капіталом. Перекладаючи „avances primitives“ і „avances annuelles“
виразами „fixed capital“ і „circulating capital“, Сміс робить крок наперед
щодо вживання слова „капітал“, поняття якого узагальнюється і стає
незалежне від особливого застосування його фізіократами до сфери
„хліборобської“; крок назад у тому, що ріжниці між „основним“ капіталом
і „обіговим“ капіталом розглядається і їх додержується як вирішальних
ріжниць.

\subsection{Адам Сміс}

\subsubsection{Загальні погляди А.~Сміса}

А.~Сміс каже в книзі І, розд. 6, стор. 42; „В усякому суспільстві
ціна кожного товару кінець-кінцем розкладається або на ту або на другу
з цих трьох частин (заробітна плата, зиск, земельна рента), або на всі
три частини; і в усякому розвиненому суспільстві всі вони три, більш
або менш, увіходять як складові частини в ціну переважної більшости
товарів“\footnote{
Щоб читача не ввів у помилку вислів „ціна переважної більшости товарів“,
наведемо витяг про те, як сам А.~Сміс розуміє цей вислів. Напр., в ціну морської
риби рента не входить, а входить лише заробітна плата й зиск; в ціну Scotch
pebbles (шотляндської ріні) входить лише заробітна плата: „В деяких частинах
Шотляндії бідняки промишляють тим, що збирають на морському березї різнокольорові
камінці, так звану шотляндську рінь. Ціна, що її платять їм за ці камінці
різьбарі, складається тільки з їхньої заробітної плати, бо ні земельна рента, ні
зиск не становлять жодної частини її“.
}; або, як сказано далі, стор. 63: „Заробітна плата, зиск і земельна
\index{ii}{0277}  %% посилання на сторінку оригінального видання
рента є \so{три первісні джерела} всякого доходу, так само,
як і всякої \so{мінової вартости}“. Далі ми розглянемо докладніше це
вчення А.~Сміса про „складові частини ціни товарів“, зглядно про „всяку
мінову вартість“. Далі він каже: „Що все це має силу для всякого
поодинокого товару, взятого окремо, то повинно воно мати силу й для
всіх товарів, разом узятих, які становлять \so{увесь річний продукт}
землі та праці кожної країни. \so{Вся ціна або мінова вартість}
цього річного продукту мусить \so{розкладатись} на ці самі три частини
та \so{розподілятись} між різними жителями країни або як \so{плата} за
їхню працю, або як \so{зиск} їхнього капіталу, або як \so{рента} з їхнього
землеволодіння“. (Кн.~II, розд., ст. 190).

Після того, як А.~Сміс і ціну всіх товарів, узятих окремо, і „всю ціну
або мінову вартість\dots{} річного продукту землі та праці кожної країни“
розклав таким чином на три джерела доходів: доходів найманого робітника,
капіталіста й земельного власника, на заробітну плату, зиск і земельну
ренту, він все ж мусить контрабандою ввести обхідним шляхом
четвертий елемент, а саме елемент капіталу. Це робиться через відрізнення
між гуртовим і чистим доходом. „\so{Гуртовий} дохід усіх жителів великої
країни охоплює \so{ввесь річний продукт} їхньої землі та їхньої праці;
чистий \so{дохід} — \so{частину}, що лишається в їхньому розпорядженні,
\so{відлічивши втрати на підтримання}, поперше, їхнього \so{основного},
а подруге, їхнього \so{поточного капіталу}, тобто
частину, що її вони можуть, не порушуючи свого капіталу, залічити
до свого споживного запасу або витратити на своє утримання, комфорт
і втіхи. Справжнє їхнє багатство теж пропорційне не їхньому гуртовому,
а чистому їхньому доходові“. (Там само, ст. 190).

На це ми зауважимо ось що:

1) А.~Сміс тут виразно розглядає тільки просту репродукцію, а не
репродукцію в поширеному маштабі, або акумуляцію; він каже лише про
видатки на підтримання (maintening) діющого капіталу. „Чистий“ дохід
дорівнює тій частині річного продукту — хоч суспільства, хоч індивідуального
капіталіста — яка може ввійти в „фонд споживання“, але розміри
цього фонду не повинні порушити діющого капіталу (encroach upon capital).
Отже, частина вартости, так індивідуального, як і суспільного продукту
не сходить ні на заробітну плату, ні на зиск або земельну ренту,
а сходить на капітал.

2) А.~Сміс ховається від своєї власної теорії за допомогою гри слів,
за допомогою розмежування між gross і net revenue — гуртовим і чистим
доходом. Поодинокий капіталіст, як і ціла кляса капіталістів, або так
звана нація, замість зужиткованого в продукції капіталу, одержує товаровий
продукт, що його вартість — її можна визначити в пропорційних частках
цього самого продукту — з одного боку, покривав витрачену капітальну
вартість, а тому становить дохід або, буквально, revenue (revenue — дієприкметник
від revenir, повертатись), однак, nota bene, являє capital-revenue або
дохід на капітал; з другого боку, маємо складові частини вартости, що
їх „розподіляється між різними жителями країни або як плату за їхню
\parbreak{}  %% абзац продовжується на наступній сторінці
