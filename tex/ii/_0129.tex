\index{ii}{0129}  %% посилання на сторінку оригінального видання

3) З цього випливає, що хоча б переважна частина авансованого
продуктивного капіталу складалася з основного капіталу, час репродукції
якого, а, значить, і час обороту, охоплює багаторічний цикл, все ж капітальна
вартість, що обертається протягом року, в наслідок повторюваних
протягом року оборотів поточного капіталу, може бути більша, ніж ціла
вартість авансованого капіталу.

Припустімо, що основний капітал \deq{} \num{80.000}\pound{ ф. стерл.}, час його репродукції
\deq{} 10 рокам, отже, \num{8.000}\pound{ ф. стерл.} щороку повертаються до своєї
грошової форми, або основний капітал робить \sfrac{1}{10} свого обороту. Хай
поточний капітал дорівнює \num{20.000}\pound{ ф. стерл.} і робить на рік п’ять оборотів.
Отже, ввесь капітал тоді дорівнює \num{100.000}\pound{ ф. стерл}. Основний
капітал, що обернувся, дорівнює \num{8.000}\pound{ ф. стерл.}, поточний капітал, що
обернувся, дорівнює $5 × \num{20.000} \deq{}$ \num{100.000}\pound{ ф. стерл}. Отже, капітал, що
обернувся протягом року \deq{} \num{108.000}\pound{ ф. стерл.}, на \num{8.000}\pound{ ф. стерл.} більший,
ніж авансований капітал. Обернулось 1 \dplus{} \sfrac{2}{25} капіталу.

4) Отже, \emph{оборот вартости} авансованого капіталу відділяється
від часу його справжньої репродукції або від часу реального обороту
його складових частин. Припустімо, що капітал в \num{4.000}\pound{ ф. стерл.} обертається
п’ять разів на рік. Тоді капітал, що обернувся, дорівнює
$5 × \num{4.000}$ \deq{} \num{20.000}\pound{ ф. стерл}. Наприкінці кожного обороту повертається,
щоб знову авансуватись, первісно авансований капітал в \num{4.000}\pound{ ф. стерл}.
Його величина не змінюється від числа тих періодів обороту, що
протягом них він знову функціонує як капітал. (Додаткову вартість
лишаємо осторонь).

Отже, в прикладі 3) згідно з припущенням, наприкінці року до рук
капіталіста повернулось: а) сума вартости з \num{20.000}\pound{ ф. стерл.}, що її він
знову витрачає на поточні складові частини капіталу, і б) сума \num{8.000}\pound{ ф.
стерл.}, що в наслідок зношування відокремилась від вартости авансованого
основного капіталу; разом з тим у продукційному процесі, як і
раніш, лишається той самий основний капітал, але вартість його
зменшилася до \num{72.000}\pound{ ф. стерл.} замість \num{80.000}\pound{ ф. стерл}. Отже, треба продовжувати
продукційний процес ще дев’ять років, поки авансований основний
капітал доживе свого віку, перестане функціонувати як продуктотворчий
\parbreak{}  %% абзац продовжується на наступній сторінці
