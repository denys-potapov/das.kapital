\parcont{}  %% абзац починається на попередній сторінці
\index{ii}{*0013}  %% посилання на сторінку оригінального видання
Ця праця, що на її значення повинен був би звернути увагу вже один
вислів: Surplus produce or capital, є памфлет на 40 сторінок, що його
Маркс видобув з непам’яті; там сказано:

„Хоч як це випадало б капіталістові (з погляду капіталіста), він
завжди може привласнювати лише додаткову працю (surplus labour) робітника,
бо робітник повинен жити“ (ст. 23). Але як робітник живе,
і тому оскільки велика може бути додаткова праця, що її привлащує
капіталіст, — це дуже відносно. „Коли вартість капіталу зменшується
не в такому відношенні, як збільшується його маса, то капіталіст буде,
витискувати з робітника продукт кожної робочої години понад той мінімум,
що з нього може існувати робітник\dots{} капіталіст може, кінець-кінцем
сказати робітникові: не треба тобі їсти хліб, бо можна прожити й на
буряках та картоплі; і ми вже дійшли цього“ (ст. 24). „Коли робітника
можна довести до такого стану, що він харчуватиметься картоплею замість
хліба, то, безперечно, правильно, що при цьому можна більше здерти з
його праці; тобто, коли, харчуючись хлібом, він повинен був на утримання
себе та своєї сім’ї \emph{залишати для себе працю понеділка й вівторка},
то, годуючись картоплею, він матиме для себе  \emph{тільки половину понеділка};
а друга половина понеділка і ввесь вівторок  \emph{звільняться або} на користь
державі або  \emph{для капіталістів}“ (ст. 26). „Безперечно (it is admitted),
що сплачувані капіталістам інтереси, чи в формі ренти, проценту або
підприємецького зиску, сплачуються з праці інших“ (ст. 23). Отже, тут
ми маємо цілком Родбертусову „ренту“, тільки замість „ренти“ сказано
„інтереси“.

Маркс робить таке до цього зауваження (рукопис „Zur Kritik“,
ст. 852): „Цей мало відомий памфлет, — а видано його тоді, коли почав
звертати на себе увагу „неймовірний латальник“ Мак Куллох, — являє
великий крок наперед порівняно з Рікардо. Додаткову вартість або
„зиск“, як зве її Рікардо (часто також додатковий продукт, surplus
produce), або interest, як називає її автор памфлету, останній визначає як
surplus labour, як додаткову працю, — працю, що її робітник виконує безплатно,
виконує понад ту кількість праці, що нею покривається вартість
його робочої сили, тобто що нею продукується еквівалент його заробітної
плати. Так само, як важливо було звести  \emph{вартість до праці}, так
само важливо було додаткову вартість (surplus value), \emph{виражену в додатковому
продукті} (surplus produce), звести до \emph{додаткової праці}
(surplus labour). \emph{Це власне сказав уже А.~Сміс, і це становить головний
момент у тому, що розвинув Рікардо}. Але ніде в них це не
висловлено в абсолютній формі й не установлено точно. Потім далі, на
стор. 859 рукопису, сказано: „А, проте, автора полонили ті економічні
категорії, що були до нього. Так само, як у Рікардо сплутування додаткової
вартости й зиску призводить до неприємних суперечностей, так
само сталось і з ним тому, що він назвав додаткову вартість
інтересом капіталу. А, проте, він стоїть вище від Рікардо тією стороною, що
він перший зводить усяку додаткову вартість до додаткової праці і, хоч зве
додаткову вартість інтересом капіталу, однак, разом з тим підкреслює,
\index{ii}{*0014}  %% посилання на сторінку оригінального видання
що interest of capital він розуміє, як загальну форму
додаткової праці на відміну від її особливих форм, ренти, проценту і
підприємецького зиску\dots{} Але назву однієї з цих особливих форм, interest,
він усе ж бере як назву загальної форми. І цього досить, щоб він знову
заплутався в економічній тарабарщині“ (в рукопису стоїть „slang“).

Цей останній пункт якнайточніше стосується й до Родбертуса. І він
також у полоні економічних категорій, що були до нього. І він зве
додаткову вартість ім’ям однієї з її перетворених підпорядкованих їй форм,
ім’ям ренти, — ренти, що її він до того ж зробив зовсім невизначеною.
Наслідок цих двох помилок був той, що він знову вдається в економічну
тарабарщину, не прокладає критично шляхів далі за Рікардо і замість
того піддається спокусі зробити з своєї недоробленої теорії, що не
вилупилася ще з шкаралупи, основу утопії, з якою він, як і завжди,
прийшов дуже пізно. Памфлет, виданий 1821 року, цілком упередив
„ренту“ Родбертуса від 1842 року.

Наш памфлет є лише крайній аванпост тієї багатої літератури, що
двадцятими роками обернула теорію вартости й додаткової вартости
Рікардо в інтересах пролетаріяту проти капіталістичної продукції, била
буржуазію її власною зброєю. Весь Оуенівський комунізм, оскільки він
бере участь в економічній полеміці, спирається на Рікардо. Але поряд
нього був ще цілий ряд письменників, що з них Маркс уже 1847~\abbr{р.} в
полеміці проти Прудона („Misère de la philosophie“, p. 49) згадує лише деяких:
Едмонда, Томпсона, Годскіна і~\abbr{т. ін.} і~\abbr{т. ін.}, „і ще чотири сторінки et
cetera“. З цих численних праць я беру одну, першу-ліпшу: „Ап Inquiry
into the Principles of the Distribution of Wealth, most conducive to Human
Happiness, by William Thompson; a new edition, London 1850“. Цей твір,
написаний 1822~\abbr{р.}, вперше видано 1827~\abbr{р.} Багатство, що його привлащують
непродуктивні кляси, тут теж усюди визначається, як відрахування з
продукту робітника, і це в досить енергійних висловах. „Повсякчасне
намагання того, що ми звемо суспільством, було в тому, щоб обманою
або умовлянням, залякуванням або примусом спонукати продуктивного
робітника виконувати працю за якомога меншу частину продукту його
власної праці“ (стор.~28). „Чому ж робітник не може одержувати абсолютно
ввесь продукт своєї роботи?“ (ст.~32). „Цю компенсацію, що її
капіталісти вимушують від продуктивних робітників під назвою земельної
ренти, або зиску, вимагають за користування землею або іншими речами\dots{}
Що всі фізичні матеріяли, на яких або за допомогою яких позбавлений
власности продуктивний робітник, що нічого не має, крім своєї здібности
продукувати, тільки й може виявити цю свою продуктивну здібність, —
що всі ці матеріяли є в посіданні інших осіб, котрих інтереси протилежні
інтересам робітника, а згода їх є передумова його діяльности, — то чи не
залежить і чи не повинно залежати від ласки цих капіталістів те, яку
частину витворів його власної праці вони побажають дати йому, в
нагороду за цю працю? (ст.~125)\dots{} порівняно з величиною утриманого
продукту, все одно, чи зветься він податком, зиском або крадіжкою\dots{}
ці відрахування“ (ст.~126) і~\abbr{т. ін.}
