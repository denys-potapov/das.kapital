
\index{ii}{*0015}  %% посилання на сторінку оригінального видання
Треба визнати, трохи соромно мені писати ці рядки. Я ще можу
припустити, що антикапіталістична англійська література двадцятих і
тридцятих років цілком невідома в Німеччині, хоч Маркс іще в „Misère
de la philosophie“ безпосередньо звертав на неї увагу і дещо з неї —
памфлет 1821~\abbr{р.}, Равенстона, Годскіна та ін. часто цитував у першому
томі „Капіталу“. Але що не лише Literatus vulgaris\footnote*{
Literatus vulgaris-вульґарний літератор. \emph{Ред.}
}, „який справді
нічого не навчився“ і з розпачу хапається за поли Родбертусового сурдута,
а також і професор з чином і гідністю, який „бундючиться
своєю вченістю“, до того забув свою клясичну політичну економію, що
серйозно закидає Марксові, ніби той у Родбертуса украв такі речі, що
їх можна знайти вже у А.~Сміса й Рікардо, — це доводить, як низько
занепала тепер офіційна політична економія.

Але тоді що ж нового сказав Маркс про додаткову вартість? Як
сталося, що Марксова теорія додаткової вартости справила таке вражіння,
як блискавка з чистого неба, і до того ж у всіх цивілізованих країнах,
тимчасом як теорії всіх його соціялістичних попередників, а між ними й
Родбертуса, не справили жодного враження?

Історія хемії може нам пояснити це на прикладі.

Як відомо, ще наприкінці XVIII століття панувала флогістична теорія,
згідно з якою суть кожного процесу горіння в тому, що від горящого
тіла відокремлюється інше, гіпотетичне тіло, абсолютна горюча речовина,
яку позначали назвою флогістон. Ця теорія була достатня, щоб пояснювати
більшість відомих тоді хемічних явищ, хоч інколи й не без деякого
натягання. Але 1774 року Прістлей відкрив відміну газу, „що був такий
чистий або такий вільний від флогістону, що порівняно з ним звичайне
повітря було вже чимось попсованим“. Він дав цьому газові назву дефлогістоване
повітря. Скоро після нього такий самий газ відкрив у Швеції
Шеле й довів, що він є в атмосфері. Він виявив також, що цей газ
зникає, коли в ньому або в звичайному повітрі спалювати якесь тіло, а
тому назвав його вогнеповітрям. „Отже, з цих даних він зробив той
висновок, що сполука, яка постає в наслідок з’єднання флогістону з
однією із складових частин повітря (тобто підчас горіння) „є не що
інше, як огонь або тепло, яке й зникає, проходячи крізь скло“\footnote{
Roscoe-Schorlemmer. Ausführliches Lehrbuch der Chemie. Braunschwelg, 1877,
I. ст. 13, 18.
}.

Прістлей, як і Шеле, описали кисень, але вони не знали, що саме
було в їхніх руках. Вони „були в полоні“ флогістичних „категорій, що
їх знайшли вони в попередників“. Елемент, що йому судилось перевернути
всі флогістичні погляди й революціонізувати хемію, марно пропадав
в їхніх руках. Але Прістлей негайно повідомив про своє відкриття
Лявуазье в Парижі, а Лявуазьє, керуючись цим новим фактом, переглянув
усю флогістичну хемію і перший відкрив, що нова відміна повітря
є новий хемічний елемент, що підчас горіння не \emph{вилучається} з горящого
тіла отой таємничий флогістон, а що цей новий елемент \emph{сполучається} з
\parbreak{}  %% абзац продовжується на наступній сторінці
