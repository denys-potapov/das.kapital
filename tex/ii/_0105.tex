\parcont{}  %% абзац починається на попередній сторінці
\index{ii}{0105}  %% посилання на сторінку оригінального видання
само й репродукція є лише засіб репродукувати авансовану вартість
як капітал, тобто як вартість, що зростає сама з себе“. (Книга І,
розд. XXI).

Три форми: І) $Г\dots{} Г'$, II) $П\dots{} П$ і III) $Т'\dots{} Т'$ відрізняються між
собою ось чим: в формі II ($П\dots{} П$) відновлення процесу, процесу репродукції,
виражено як дійсне, а в формі І лише як можливе. Але обидві ці
форми відрізняються від форми III тим, що авансована капітальна вартість
— хоч її авансовано як гроші, хоч в вигляді речових елементів продукції
— становить вихідний пункт, а тому й пункт повороту. В $Г\dots{} Г'$ поворот
є $Г' \deq{} Г \dplus{} г$. Коли процес відновлюється знову в тих самих розмірах,
то $Г$ знову становить вихідний пункт, а $г$ не входить в цей процес і лише
показує нам, що $Г$ зросло своєю вартістю як капітал і тому створило додаткову
вартість $г$, але відштовхнуло її від себе. В формі $П\dots{} П$ капітальна
вартість, авансована в формі $П$, елементів продукції, знову таки
становить вихідний пункт. Ця форма має в собі й зростання цієї вартости.
Коли відбувається проста репродукція, то та сама капітальна вартість
в тій самій формі $П$ знову починає свій процес. Коли відбувається акумуляція,
то тепер процес починає $П'$ (що величиною вартости дорівнює
$Г' \deq{} Т'$), як збільшена капітальна вартість. Але процес починається знову
авансованою капітальною вартістю в початковій формі, хоч і капітальною
вартістю більшою, ніж раніш. Навпаки, в формі III капітальна вартість
починає процес не як авансована, але як уже виросла, як усе багатство,
що перебуває в формі товарів, і що лише деяка частина його являє
авансовану капітальну вартість. Остання форма важлива для третього
відділу, де рух поодиноких капіталів береться в зв’язку з рухом сукупного
суспільного капіталу. Але, навпаки, з неї не можна користатись,
коли досліджується оборот капіталу, що завжди починається авансуванням
капітальної вартости, чи то у формі грошей, чи то у формі товару, і
який завжди зумовлює, що капітальна вартість, яка чинить оборот, повертається
в тій формі, що в ній її авансовано. З кругобігів І і II
треба триматися першого, коли мають на увазі переважно той вплив, що
його справляє оборот на утворення додаткової вартости; другого — коли
мають на увазі вплив обороту на утворення продукту.

Як мало економісти відрізняли різні форми кругобігів, так само мало
вони розглядали ці різні форми кругобігів відокремлено щодо обороту
капіталу. Звичайно береться форму $Г\dots{} Г'$, бо вона панує над поодиноким
капіталістом і служить йому в його розрахунках навіть тоді, коли
гроші становлять вихідний пункт лише в формі рахункових грошей. Інші
беруть за вихідний пункт витрати в формі елементів продукції, поки
не настане поворот, при цьому про форму повороту — чи буде цей поворот
в товарі, чи в грошах — у них немає й мови. Напр.: „Економічний
цикл\dots{} тобто ввесь перебіг продукції від часу, коли зроблено витрати,
до часу, коли настає поворот. В сільському господарстві час засіву є
початок економічного циклу, а жнива — закінчення“. (Economic Cycle\dots{}
the whole course of production, from the time that outlays are made till
returns are received. In agriculture seedtime is its commencement, and
\parbreak{}  %% абзац продовжується на наступній сторінці
