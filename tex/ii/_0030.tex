
\index{ii}{0030}  %% посилання на сторінку оригінального видання
$Г — Г'$ стає особливою формою кругобігу промислового капіталу,
оскільки капітал, що заново виступає, авансується спочатку як гроші, а потім
у тій самій формі береться назад, — хоч це буде підчас переходу з однієї
галузі підприємств до іншої, хоч підчас виходу промислового капіталу
з підприємства. Ця форма має в собі й функціонування як капіталу
тієї додаткової вартости, що її авансується спочатку в грошовій формі,
і помітно це якнайяскравіше, коли вона функціонує в іншому підприємстві,
а не в тому, відки вона постає. $Г — Г'$ може бути першим кругобігом
певного капіталу, і воно може бути останнім; воно може вважатись за
форму цілого суспільного капіталу; це — форма капіталу, що його вкладається
вперше, все одно, хоч він є новоакумульований капітал у грошовій
формі, хоч старий капітал, цілком перетворений на гроші для того,
щоб перенести його з однієї галузі продукції до іншої.

Як форму, постійно властиву всім кругобігам, грошовий капітал пророблює
цей кругобіг саме для тієї частини капіталу, яка утворює додаткову
вартість для змінного капіталу. Нормальна форма авансування
заробітної плати є виплата грішми; цей процес мусить постійно відновлюватися
з короткими промежками, бо робітник перебивається від
получки до получки. Тому капіталіст мусить завжди протистояти робітникові
як грошовий капіталіст, а його капітал як грошовий капітал. Тут
не може відбуватися безпосереднього або посереднього вирівнювання
виплат (так що більшість грошового капіталу справді фігурує лише
в формі товарів, гроші — лише в формі розрахункових грошей, а готівка,
кінець-кінцем — лише для вирівнювання балансів), як це буває при купівлі
засобів продукції та продажу продуктивних товарів. З другого боку,
частину тієї додаткової вартости, що постає із змінного капіталу, капіталіст
витрачає на своє особисте споживання, що належить до сфери
роздрібної торговлі, і цю частину він, кінець-кінцем, завжди витрачає
готівкою, в грошовій формі додаткової вартости. Чи велика, чи мала ця
частина додаткової вартости, це не змінює справи. Змінний капітал
знову й знову виступає як грошовий капітал, витрачуваний на заробітну
плату ($Г — Р$), а $г$ — як додаткова вартість, витрачувана на покриття
особистих потреб капіталіста. Отже, $Г$ як авансована змінна капітальна
вартість, і $г$ як її приріст, обидва неодмінно зберігаються в грошовій
формі, щоб можна було їх в цій формі витрачати.

Формула $Г — Т\dots{} П\dots{} Т' — Г'$, з своїм результатом $Г' \deq{} Г \dplus{} г$, обманна
своєю формою, має ілюзорний характер, який випливає з того, що
авансована й виросла вартість існують тут у своїй еквівалентній формі — в грошах. У цій формі
наголос є не на зростанні вартости, а на
\emph{грошовій формі} цього процесу, на тому, що з циркуляції, кінець-кінцем,
вилучається більш вартости в грошовій формі, ніж первісно було авансовано, — отже, на збільшенні
маси золота й срібла, що належить капіталістові.
Так звана монетарна система є просто вираз іраціональної форми $Г — Т — Г'$,
руху, що відбувається виключно в сфері циркуляції; тому обидва акти:
1) $Г — Т$ і 2) $Т — Г'$ вона може пояснити тільки тим, що $Т$ в другому
акті продається дорожче проти своєї вартости, а тому й більше грошей
\parbreak{}  %% абзац продовжується на наступній сторінці
