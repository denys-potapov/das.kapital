\parcont{}  %% абзац починається на попередній сторінці
\index{ii}{0265}  %% посилання на сторінку оригінального видання
І це повторюється постійно. Отже, сума в $100\pound{ ф. стерл.} × x$ ніколи не
може дати робітничій клясі змоги купити частину продукту, яка репрезентує
сталий капітал, не кажучи вже про ту частину, яка репрезентує додаткову
вартість кляси капіталістів. Робітники на $100\pound{ ф. стерл.} × x$ завжди можуть
купити тільки ту частину вартости суспільного продукту, яка дорівнює
тій частині вартості, що репрезентує вартість авансованого змінного
капіталу.

Лишаючи осторонь той випадок, коли ця всебічна грошова акумуляція
не виражає нічого іншого, крім розподілу в тому або іншому відношенні
додатково довезеного благородного металю між різними поодинокими
капіталістами, — отже, лишаючи осторонь цей випадок, яким чином може
акумулювати гроші ціла кляса капіталістів?

Всі вони мусили б продавати частину свого продукту, нічого не купуючи
натомість. Немає нічого таємничого в тому, що всі вони мають
певний грошовий фонд, який вони пускають в циркуляцію як засіб циркуляції
для свого споживання, при цьому до кожного з них зворотно припливає
з циркуляції певна частина цього фонду. Але в такому разі такий фонд
існує саме як фонд циркуляції, що утворився в наслідок перетворення
на гроші додаткової вартости, але зовсім не як лятентний грошовий
капітал.

Коли розглядати справу так, як вона відбувається в дійсності, то лятентний
грошовий капітал, що його нагромаджується для пізнішого вжитку,
складається з:

1) Депозитів у банках; при цьому сума грошей, що нею в дійсності
порядкують банки, є порівняно незначна. Грошовий капітал тут нагромаджується
лише номінально, але що в дійстності нагромаджується тут,
так це грошові вимоги, які лише тому можна перетворити на гроші
(якщо тільки можна перетворити), що установлюється рівновага між
зворотними грошовими вимогами і вкладами в банк. А те, що є в банку
як гроші, являє лише порівняно невелику суму.

2) З державних паперів. Вони взагалі не є капітал, а лише боргові
вимоги на частину річного продукту нації.

3) З акцій. Оскільки це не шахрайство, вони є титули власности на
дійсний капітал, належний товариству, і посвідки на одержання додаткової
вартости, яку щорічно дає цей капітал.

В усіх цих випадках немає жодного нагромадження грошей: те, що
на одному боці виступає як нагромадження грошового капіталу, виступає
на другому боці як постійне справжнє витрачання грошей. Хоч витрачає
гроші та особа, якій вони належать, хоч інша, її винуватець, це зовсім
не змінює справи.

На основі капіталістичного способу продукції утворення скарбу, як
таке, ніколи не є мета, а результат або застою в циркуляції — і тоді,
більші, ніж звичайно, маси грошей набирають форми скарбу, — або нагромаджень,
зумовлених оборотом, або нарешті: утворення скарбу є лише
утворення грошового капіталу — покищо в лятентній формі, — призначеного
функціонувати як продуктивний капітал.
