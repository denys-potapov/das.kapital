\parcont{}  %% абзац починається на попередній сторінці
\index{ii}{0086}  %% посилання на сторінку оригінального видання
але не для того, щоб утворити вартість, а для того, щоб перетворити
вартість з однієї форми на іншу, при чому справа не змінюється від
взаємних намагань при цій нагоді присвоїти собі надлишкову кількість
вартости. Ця робота, зловмисно перебільшувана обома сторонами, так само не
утворює вартости, як і робота, витрачена на судовому процесі, не
збільшує вартости предмету позову. З цією роботою, — яка є неодмінний
момент капіталістичного продукційного процесу в його сукупності, коли
він містить у собі циркуляцію або сам міститься в ній, — справа стоїть
так само, як, напр., з роботою горіння речовини, що нею користуються для
видобування тепла. Ця робота горіння не створює тепла, хоч вона є
неодмінний момент процесу горіння. Щоб, напр., зужити вугілля як
паливо, я мушу сполучити його з киснем і для цього перевести його з
твердого стану в газуватий (бо у вуглекислому газі, наслідку горіння,
вугілля перебуває в стані газуватому), отже, я мушу перевести зміну
фізичної форми буття або стану. Відділення молекуль вуглецю, сполучених
в одне тверде тіло, і розпад самих молекуль вуглецю на окремі атоми,
мусить відбутись раніш, ніж постане нова сполука, а для цього треба
прикласти деяку силу, що, отже, не перетворюється на тепло, а
відбирається від нього. Тому, коли товаровласники не є капіталісти, а
самостійні безпосередні продуценти, то час, витрачений від них на купівлю
й продаж, є одбава з їхнього робочого часу. Ось чому вони завжди
(і в старовину, і в середні віки) дбали про те, щоб такі операції відкладати
на святкові дні.

Розміри, що їх доходить перетворення товарів у руках капіталістів, не
можуть звичайно перетворити цю роботу, яка не утворює жодної вартости,
а впосереднює лише зміну форм вартости, на роботу, що утворює вартість.
Чудо такого перетворення так само мало може постати в наслідок
перекладання, тобто в наслідок того, що промислові капіталісти замість
самим виконувати оту „роботу горіння“, роблять з неї виключне зайняття
оплачуваних ними третіх осіб. Ці треті особи, звичайно, не дають їм
своєї робочої сили заради краси їхніх очей. Для збирача орендної плати,
що служить у якогобудь землевласника, або для банківського службовця
так само байдуже, що їхня праця ні на шеляг не збільшує величини
вартости ані ренти, ані зливків золота, що їх переноситься в мішках до
іншого банку\footnote{
Заведене в дужки взято з примітки наприкінці рукопису VIII.
}].
\label{original-86}

Для капіталіста, що примушує інших робити на себе, купівля й
продаж стають головною функцією. Що він привлащує продукт багатьох
у широкому суспільному маштабі, то в такому самому маштабі має він
продавати цей продукт, а потім знову зворотно перетворювати з грошей
на елементи продукції. Але, як і раніше, час купівлі, та продажу не
утворює жодної вартости. Ілюзія постає тут у наслідок функції купецького
капіталу. Але покищо, не розглядаючи цього ближче, само собою
зрозуміло таке. Коли якась функція, що сама собою непродуктивна, але
становить доконечний момент репродукції, в наслідок поділу праці перетворюється
\index{ii}{0087}  %% посилання на сторінку оригінального видання
з бічної функції багатьох на виключну функцію небагатьох,
на їхнє особливе зайняття, то від цього самий характер функції не зміниться.
Можливо, що купець (тут розглядуваний просто як аґент перетворення
форми товарів, лише як покупець і продавець) своїми операціями
скорочує для \emph{багатьох} продуцентів час, витрачуваний на купівлю
й продаж. Тоді його треба розглядати як машину, що зменшує некорисну
витрату сили, або допомагає врятувати час для продукції\footnote{
Торгові витрати, хоч і доконечні, все ж повинно вважати за обтяжливі видатки“.
(Les frais de commerce, quoique nécessaires, doivent être regardés comme une
dépense onéreuse. — Quesnay, „Analyse du Tableau Economique“, y Daire, Physiocrates,
1-е partie, Paris, 1846, p. 71). — Згідно з Кене, „зиск“, що його дає конкуренція
між торговцями, примушуючи їх зробити поступку з своєї винагороди або бариша\dots{},
власне кажучи, є лише усунення втрати для продавця з перших
рук і для покупця-споживача. Але усунення втрати, що її спричиняють торгові
витрати, не є реальний продукт, або збільшення багатства в наслідок торговлі,
розглядуваної в собі самій просто як обмін, незалежно від транспортових витрат
або разом з цими витратами“ (à mettre leur rétribution ou leur gain au rabais\dots{}
n’est rigoureusement parlant, qu’une, privation de perte pour le vendeur de la première
main et pour l’acheteur-consommateur. Or, une privation de perte sur les frais du
commerce n'est pas un produit réel ou un accroît de richesses obtenu par le commerce,
considérée en lui-même simplement comme échange, indépendemment des frais de transport, ou envisagé
conjointement avec les frais de transport) (145, 146 стор.).

„Торгові витрати завжди оплачується коштом продавця продуктів, що був би
одержував усю ціну, яку дають за них покупці, коли б не було жодних витрат
на посередництво“ („Les frais du commerce sont toujours payés aux dépens des
vendeurs des productions qui jouiraient de tout le prix qu’en payent les acheteurs,
s’il n’y avait point de frais intermédiaires) (p. 163). Власники й продуценти є
„salariants“ — ті, хто оплачують; купці — „salariés“, оплачувані, наймані (стор. 164)
(Quesnay, Problèmes économiques, у Daire, Physiocrates, 1-е partie, Paris, 1846).
}.

Щоб спростити справу (бо ми лише пізніше розглядатимемо купця
як капіталіста і купецький капітал), ми припустимо, що аґент купівлі та
продажу є людина, яка продає свою працю. Він витрачає свою робочу
силу і свій робочий час на ці операції $Т — Г$ і $Г — Т$. Він живе з цього
так само, як, напр., інший живе з прядіння або готування пілюль. Він
виконує доконечну функцію, бо самий процес репродукції має в собі
непродуктивні функції. Він працює так само, як і інший, але зміст його
праці не утворює ні вартости, ні продукту. Він сам належить до faux
frais\footnote*{
Faux frais (франц.) — фалшиві витрати, тобто непродуктивні витрати. \emph{Ред.}
} продукції. Користь від нього не в тому, що він перетворює
непродуктивну функцію на продуктивну або непродуктивну працю на
продуктивну. Було б чудо, коли б таке перетворення сталось у наслідок
перекладання функції від однієї особи на іншу. Скоріше, він дає користь
тим, що меншу частину робочої сили й робочого часу суспільства
зв’язується цією непродуктивною функцією. Навіть більше. Припустімо,
що він простий найманий робітник, хоч і краще оплачуваний.
Хоч би    як оплачувалось його працю, все ж частину свого часу
він, як    найманий робітник, працює    задурно. Можливо, він
одержує щодня вартість, спродуковану протягом вісьмох робочих годин, а
працює протягом десятьох годин. Дві години додаткової праці, виконуваної
\index{ii}{0088}  %% посилання на сторінку оригінального видання
ним, так само не продукують вартости, як і його вісім годин
доконечної праці, хоч у наслідок цих останніх до нього переходить
частина суспільного продукту. Поперше, з суспільного погляду, протягом
усіх десятьох годин робочу силу використовується, як і раніше, на цю
просту функцію циркуляції. Її не можна вживати на що інше, не
можна вживати на продуктивну працю. Подруге, суспільство не оплачує
цих двох годин додаткової праці, хоч їх і витратила особа, що
працювала протягом цього часу. Суспільство не одержує через це жодного
додаткового продукту або вартости. Але витрати циркуляції, що їх
ця особа репрезентує, зменшуються на одну п’яту частину: з десятьох
годин до вісьмох. Суспільство не виплачує жодного еквіваленту за п’яту
частину того активного часу циркуляції, що його агентом є ця особа.
А коли це капіталіст, що вживає таких агентів, то неоплачені дві години
зменшують ті витрати циркуляції \emph{його} капіталу, які становлять одбаву з його
прибутків. Для нього це — позитивний виграш, бо неґативні межі зростання
вартости його капіталу вужчають. Поки дрібні самостійні товаропродуценти
витрачають частину свого власного часу на купівлю й продаж,
він являє або час, витрачуваний у промежках їхньої продуктивної
діяльности, або час, що віднімається від їхнього часу продукції.

За всіх обставин час, витрачений на це, є витрати циркуляції, що
нічого не додають до перетворених вартостей. Це є витрати, — потрібні
для того, щоб перетворити вартості з товарової форми на грошову.
Оскільки капіталістичний товаропродуцент є аґент циркуляції, він відрізняється
від безпосереднього товаропродуцента лише тим, що продає й
купує в ширших розмірах, а тому й функціонує як аґент циркуляції в
ширшому маштабі. Але коли розмір його підприємства примушує його
або дозволяє йому купувати (наймати) власних аґентів циркуляції, як
найманих робітників, то суть справи від цього не змінюється. Робочу
силу й робочий час треба до певного ступеня витратити на процес циркуляції
(оскільки він є просте перетворення форми). Однак тепер ця витрата
являє додаткову витрату капіталу; частину змінного капіталу доводиться
витрачати на закуп цієї робочої сили, що функціонує лише в
циркуляції. Таке авансування капіталу не утворює ні продукту, ні
вартости. Воно зменшує pro tanto\footnote*{
Відповідно. \emph{Ред.}
} розміри, що в них авансований
капітал функціонує продуктивно. Це те саме, ніби частину продукту
перетворили на машину, що купувала б і продавала решту продукту. Ця
машина зумовлює одбаву з продукту. Вона не співдіє в процесі продукції,
хоч може зменшити робочу силу тощо, витрачувану на циркуляцію.
Вона становить лише частину витрат циркуляції.

\subsubsection{Бухгальтерія}

Поряд справжніх купівель і продажів робочий час витрачається на
ведення книг, куди, крім того, входить і зречевлена праця: пера, чорнила,
папір, бюрко, витрати на контору. Отже, на цю функцію витрачається, з
\parbreak{}  %% абзац продовжується на наступній сторінці
