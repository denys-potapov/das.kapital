\parcont{}  %% абзац починається на попередній сторінці
\index{i}{0104}  %% посилання на сторінку оригінального видання
квартер збіжжя за 3\pound{ фунти стерлінґів} і за ці 3\pound{ фунти стерлінґів}
купую одяг, то для мене ці 3\pound{ фунти стерлінґів} остаточно витрачені.
Я вже не маю ніякого відношення до них. Вони вже належать
торговцеві одягом. А коли я продам другий квартер збіжжя,
то гроші знов повернуться до мене, але не в наслідок першої
оборудки, а лише в наслідок її повторення. Вони знову віддаляться
від мене, скоро тільки я другу оборудку доведу до кінця
і знов куплю. Отже, в циркуляції $Т — Г — Т$ витрата грошей
не має ніякого відношення до їхнього зворотного припливу. Навпаки,
в $Г — Т — Г$ зворотний приплив грошей зумовлюється
характером самої їхньої витрати. Без цього зворотного припливу
операція не вдається; процес переривається і не є ще закінчений,
бо бракує другої його фази, продажу, що доповнює й завершує
купівлю.

Кругобіг $Т — Г — Т$ має за вихідний пункт якийсь товар і
за кінцевий пункт якийсь інший товар, що виходить із циркуляції
і ввіходить у споживання. Тому споживання, задоволення
потреб, одне слово, споживна вартість є його кінцева мета.
Навпаки, кругобіг $Г — Т — Г$ має за вихідний пункт грошовий
полюс, і кінець-кінцем він повертається назад до того самого
полюса. Тому його движний мотив і мета, що його визначає, є
сама мінова вартість.

У простій товаровій циркуляції обидва крайні пункти мають
ту саму економічну форму. Обидва вони є товари. Вони є також
товари однакової величини вартости. Але вони є якісно різні
споживні вартості, приміром, збіжжя й одяг. Обмін продуктів,
обмін різних речовин, що в них виражається суспільна праця,
становить тут зміст руху. Інша справа в циркуляції $Г — Т — Г$.
На перший погляд вона здається беззмістовною через свою тавтологічність.
Обидва полюси мають таку саму економічну форму.
Вони обидва є гроші, отже, вони не є якісно відмінні споживні
вартості, бо гроші — це саме й є така перетворена форма товарів,
у якій згасають усі їхні особливі споживні вартості. Спочатку
обміняти 100\pound{ фунтів стерлінґів} на бавовну, а потім знову цю саму
бавовну обміняти на 100\pound{ фунтів стерлінґів}, тобто манівцями
гроші на гроші, те саме на те саме, — це видається так само безцільною,
як і безглуздою операцією\footnote{
«Гроші не обмінюють на гроші» — вигукує Мерсьє де ля Рів’єр
на адресу меркантилістів. («L’Ordre naturel et essentiel des sociétés politiques»,
Physiocrates, éd. Daire, p. 486). В одному творі, де мова йде
ex professo про «торговлю» й «спекуляцію», ми читаємо: «Всяка торговля
сходить на обмін різнорідних речей; і користь [для купця?] випливає
саме з цієї різнорідности. Обмін одного фунта хліба на один фунт хліба\dots{}
був би без жодної користи\dots{} звідси корисний контраст між торговлею
і грою, яка є лише обмін грошей на гроші». (\emph{Th.~Corbet}: «An Inquiry
into the Causes and Modes of the Wealth of Individuals; or the Principles
of Trade and Speculation explained», London 1841, p. 5). Хоч Корбет і не
бачить, що $Г — Г$, обмін грошей на гроші є форма циркуляції, характеристична
не лише для торговельного капіталу, але й для кожного капіталу,
однак він принаймні признає, що ця форма особливого роду торговлі,
спекуляції, є форма гри; але з’являється Мак Куллох і відкриває, що
\parbreak{}
купівля задля продажу є спекуляція, і таким чином відпадає ріжниця
між спекуляцією й торговлею. «Всяка операція, коли одна особа купує
продукт із наміром продати його, є в дійсності спекуляція» («Every
transaction in which an individual buys produce in order to sell it again,
is, in fact, a speculation»). (\emph{Mac Culloch}: «А Dictionary practical etc.
of Commerce», London 1847, p. 1009). Куди наївнішим є Пінто, цей Піндар
амстердамської біржі: «Торговля — це гра (це речення він запозичає
у Льокка) і, звичайно, граючи з тим, хто нічого не має, не можна
виграти. Коли б хто протягом довгого часу постійно в усіх вигравав,
то він мусив би добровільно повернути більшу частину свого зиску,
щоб знов почати гру». («Le commerce est un jeu, et ce n’est pas avec des
gueux qu’on peut gagner. Si l’on gagnait longtemps en tout avec tous, il
taudrait rendre de bon accord les plus grandes parties du profit, pour
recommencer le jeu»). (Pinto: «Traité de la Circulation et du Crédit».
Amsterdam 1771, p. 231).
}. Певна сума грошей може
\index{i}{0105}  %% посилання на сторінку оригінального видання
взагалі відрізнятися від іншої суми грошей лише своєю величиною.
Отже, процес $Г — Т — Г$ завдячує свій зміст не якісній
ріжниці своїх полюсів, бо обоє вони є гроші, а лише їхній кількісній
ріжниці. Кінець-кінцем із циркуляції витягається більше
грошей, ніж спочатку туди їх подано. Бавовну, куплену за 100\pound{ фунтів стерлінґів}, знову продається, приміром, за 100 \dplus{} 10\pound{ фунтів
стерлінґів}, або за 110\pound{ фунтів стерлінґів}. Отже, повна форма
цього процесу є $Г — Т — Г'$, де $Г' \deq{} Г \dplus{} ΔГ$, тобто дорівнює
первісно авансованій грошовій сумі плюс приріст. Цей приріст,
або надлишок понад первісну вартість, я називаю додатковою
вартістю (surplus value). Отже, первісно авансована вартість не
лише зберігається в циркуляції, але в ній вона ще змінює величину
своєї вартости, долучає до себе якусь додаткову вартість,
або зростає у своїй вартості. І цей рух перетворює її на
капітал.

Правда, можливо також, що у формі $Т — Г — Т$ обидва полюси
$T$, $T$, приміром, збіжжя й одяг, є кількісно різні величини
вартости. Селянин може продати своє збіжжя понад його вартість
або купити одяг нижче від його вартости. З другого боку, його
може обдурити продавець одягу. Однак, така ріжниця вартости
лишається чисто випадковою для цієї форми циркуляції. Ця форма,
цілком протилежно до процесу $Г — Т — Г$, зовсім не втрачає
значення й рації, коли її обидва полюси, приміром, збіжжя й
одяг, є еквіваленти. Тут рівність їхньої вартости є скорше умова
нормального перебігу процесу.

Повторення або поновлення продажу задля купівлі находить,
як і сам цей процес, міру й доцільність у кінцевій меті, що лежить
поза межами цього процесу, — у споживанні, задоволенні певних
потреб. Навпаки, в купівлі задля продажу початок і кінець
є те саме, гроші, мінова вартість, і вже через це рух цей є безкраїй.
Правда, з $Г$ стало $Г \dplus{} ΔГ$, з 100\pound{ фунтів стерлінґів} — 100 \dplus{} 10\pound{ фунтів
стерлінґів}. Але розглянуті лише щодо якости ці 110\pound{ фунтів
стерлінґів} є те саме, що й 100\pound{ фунтів стерлінґів}, а саме гроші.
Розглянуті щодо кількости 110\pound{ фунтів стерлінґів} є обмежена сума
вартости, так само як і 100\pound{ фунтів стерлінґів}. Коли б ці 110\pound{ фунтів
стерлінґів} було витрачено як гроші, то вони перестали б
\parbreak{}  %% абзац продовжується на наступній сторінці
